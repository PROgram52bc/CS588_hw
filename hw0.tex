\documentclass[12pt]{article}
\usepackage{amsmath, amssymb}

\title{CS 577: Homework 0}
\author{David Deng}

\begin{document}

\maketitle

\section*{Exercise 0.10}

Suppose you repeatedly flip a coin that is heads with fixed probability 
$p \in (0,1)$.

\begin{enumerate}
  \item What is the expected number of coin flips until you obtain one heads? 
  Prove your answer.
  
  \vspace{4cm} % space for your solution
  
  \item What is the expected number of coin flips until you obtain two heads? 
  Prove your answer.
  
  \vspace{4cm} % space for your solution
  
  \item For general $k \in \mathbb{N}$, what is the expected number of coin 
  tosses until you obtain $k$ heads? Prove your answer.
  
  \vspace{6cm} % space for your solution
\end{enumerate}

\section*{Exercise 1.2}

Recall the quick-select algorithm introduced in Section 1.1.3. 
The goal of this exercise is to prove that quick-select takes $O(n)$ time in expectation.
Below we present two different approaches which offer two different perspectives.
Both analyses should use linearity of expectation and we ask you to point this out explicitly.

\begin{enumerate}
  \item \textbf{Approach 1.} Analyze quick-select similarly to quick-sort, based on the sum
  of indicators $X_{ij}$.
  One approach is to reduce to a separate analysis for each of the following 4
  classes of pairs:
  \begin{enumerate}
    \item $X_{ij}$ where $i < j < k$,
    \item $X_{ij}$ where $i < k < j$,
    \item $X_{ij}$ where $k < i < j$, and
    \item $X_{ij}$ where either $i = k$ or $j = k$.
  \end{enumerate}
  For each case, show that the expected sum is $O(n)$.
  
  \vspace{6cm} % space for your solution
  
  \item \textbf{Approach 2.} The following approach can be interpreted as a randomized divide
  and conquer argument. We are arguing that with constant probability, we
  decrease the input by a constant factor, from which the fast (expected) running
  time follows.
  \begin{enumerate}
    \item Consider again quick-select. Consider a single iteration where we pick a
    pivot uniformly at random and throw out some elements. Prove that with
    some constant probability $p$, we either sample the $k$th element or throw
    out at least $1/4$ of the remaining elements.
    
    \vspace{3cm}
    
    \item For each integer $i$, prove that the expected number of iterations (i.e.,
    rounds of choosing a pivot) of quick-select, where the number of elements
    remaining is in the range $\bigl[(4/3)^i, (4/3)^{i+1}\bigr)$, is $O(1)$.
    
    \vspace{3cm}
    
    \item Fix an integer $i$, and consider the amount of time spent by quick-select
    while the number of elements remaining is greater than $(4/3)^{i-1}$ and at
    most $(4/3)^i$. Show that the expected amount of time is $\leq O((4/3)^i)$.
    
    \vspace{3cm}
    
    \item Finally, use the preceding part to show that the expected running time of
    quick-select is $O(n)$.
    
    \vspace{3cm}
  \end{enumerate}
\end{enumerate}

\section*{Exercise 1.4}

This exercise is about a simple randomized algorithm for verifying
matrix multiplication. Suppose we have three $n \times n$ matrices $A, B, C$. 
We want to verify if $AB = C$. Of course one could compute the product $AB$ 
and compare it entrywise to $C$. But multiplying matrices is slow: 
the straightforward approach takes $O(n^3)$ time and there are 
(more theoretical) algorithms with running time roughly $O(n^{2.37}\ldots)$. 
We want to test if $AB = C$ in closer to $n^2$ time.

The algorithm we analyze is very simple. Select a point 
$x \in \{0,1\}^n$ uniformly at random. (That is, each $x_i \in \{0,1\}$ 
is an independently sampled bit.) Compute $A(Bx)$ and $Cx$, and compare 
their entries. (Note that it is much faster to compute $A(Bx)$ than $AB$.) 
If they are unequal, then certainly $AB \neq C$ and we output false. 
Otherwise we output true. Note that the algorithm is always correct if $AB = C$, 
but could be wrong when $AB \neq C$. We will show that if $AB \neq C$, 
the algorithm is correct with probability at least $1/2$.

\begin{enumerate}
  \item Let $y \in \mathbb{R}^n$ be a fixed nonzero vector, and let 
  $x \in \{0,1\}^n$ be drawn uniformly at random. 
  Show that $\langle x, y \rangle \; \overset{\text{def}}{=} \; 
  \sum_{i=1}^n x_i y_i \neq 0$ with probability at least $1/2$.
  
  \vspace{5cm}
  
  \item Use the preceding result to show that if $AB \neq C$, 
  then with probability at least $1/2$, $ABx \neq Cx$.
  
  \vspace{4cm}
  
  \item Suppose we want to decrease our probability of error to (say) $1/n^2$. 
  Based on the algorithm above, design and analyze a fast randomized algorithm 
  with the following guarantees:
  \begin{itemize}
    \item If $AB = C$, then it always reports that $AB = C$.
    \item If $AB \neq C$, then with probability of error at most $1/n^2$, 
    it reports that $AB \neq C$.
  \end{itemize}
  Your analysis should include the running time as well.
  
  \vspace{6cm}
\end{enumerate}

\end{document}