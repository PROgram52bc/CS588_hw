\documentclass[12pt]{article}
\usepackage{amsmath, amssymb}
\usepackage{geometry,enumitem}
\usepackage{xcolor}

\newcommand{\pts}[1]{\textbf{#1 pts}}
\newcommand{\OK}{\(\checkmark\)}
\newcommand{\so}{\(\Rightarrow\)}

\title{CS 588: Homework 0}
\author{David Deng}

\begin{document}

\maketitle

\section*{Exercise 0.10}

Suppose you repeatedly flip a coin that is heads with fixed probability 
$p \in (0,1)$.

\begin{enumerate}
  \item What is the expected number of coin flips until you obtain one heads? 
  Prove your answer.

  \vspace{1cm} % space for your solution

  Let $p_i$ be the probability of getting the \textit{first} heads on the
  i$^{th}$ flip, then we must have gotton tails on the first $i-1$ flips and heads
  on the i$^{th}$ flip.
  
  Thus,

  \[
    p_{\mathbf{1}i} = (1-p)^{i-1} p
  \]

  Since we get the first heads on the i$^{th}$ flip, then the expected number of
  flips until obtaining one heads is the distributed sum of the probability of
  obtaining the first heads on any flip.

  \begin{align*}
  \mathbb{E}[X_1] 
    &= \sum_{i=1}^{\infty} i p_{\mathbf{1}i} \\
    &= \sum_{i=1}^{\infty} i (1-p)^{i-1} p \\
    &= p \sum_{i=1}^{\infty} i (1-p)^{i-1} \\
    &= p \sum_{i=0}^{\infty} (i+1)(1-p)^i \\
    &= p \sum_{i=0}^{\infty} i(1-p)^i + (1-p)^i \\
    &= p \bigl(\sum_{i=0}^{\infty} i(1-p)^i + \sum_{i=0}^{\infty} (1-p)^i\bigr) \\
    &= p \bigl(\frac{1-p}{p^2} + \frac{1}{p}\bigr) \quad \text{(by formula for geometric series and its derivative)}\\
    &= \frac{1-p}{p} + 1 \\
    &= \frac{1}{p}
  \end{align*}


  \vspace{1cm} % space for your solution
  
  \item What is the expected number of coin flips until you obtain two heads? 
  Prove your answer.

  \vspace{1cm} % space for your solution

  \textbf{Method 1:}
  The expected number of flips until obtaining the first heads is $G_1 = 1/p$.
  The expected number of flips after obtaining the first heads until obtaining the second heads is also $G_2 = 1/p$.
  Therefore, the expected number of flips until obtaining two heads is the sum of the two:
  
  $\mathbb{E}[X_2] = \mathbb{E}[X_1 + X_1] = \mathbb{E}[X_1] + \mathbb{E}[X_1] = 2/p$.

  \textbf{Method 2 (incomplete):}

  In order to have two heads on the i$^{th}$ flip, we must have had exactly one
  heads on the first i$-1$ flips, and another head on the i$^{th}$ flip. 

  The probability of the former is

  \[ \binom{i-1}{1} p(1-p)^{i-2} = (i-1) p(1-p)^{i-2} \]

  The probability of the latter is simply $p$.

  Since the two events are independent, the joint probability is the product of the two:

  \[ p_{\mathbf{2}i} = \binom{i-1}{1} p(1-p)^{i-2} \cdot p = (i-1) p^2(1-p)^{i-2} \]

  Then the expected number of flips until obtaining two heads is the distributed
  sum of the probability of obtaining the second heads on any flip.

  \[ \mathbb{E}[X] = \sum_{i=2}^{\infty} i p_{\mathbf{2}i} = \sum_{i=2}^{\infty} i (i-1) p^2(1-p)^{i-2} = p^2 \sum_{i=2}^{\infty} i (i-1) (1-p)^{i-2} \]

  {\color{red} TODO: finish this way of showing it. Use the formula for the
  second derivative of the geometric series.}


  \item For general $k \in \mathbb{N}$, what is the expected number of coin 
  tosses until you obtain $k$ heads? Prove your answer.

  \vspace{1cm} % space for your solution

  \textbf{Method 1:}

  Let $G_i$ be the expected number of flips after obtaining the $i-1^{th}$ heads
  until obtaining the $i^{th}$ heads.

  The expected number of flips for $G_i$ is the same as that for $G_1$, since the coin
  flips are independent and identically distributed.

  Therefore, the expected number of flips until obtaining $k$ heads is
  $\mathbb{E}[\sum_{i=1}^{\infty} G_i] = \sum_{i=1}^{\infty} \mathbb{E}[G_i] = \sum_{i=1}^{\infty} 1/p = k/p$.

  \textbf{Method 2:}
  

  In general, in order to obtain $k$ heads on the i$^{th}$ flip, we must have had
  $k-1$ heads on the first i$-1$ flips, and another head on the i$^{th}$ flip.

  The probability of the former is

  \[ \binom{i-1}{k-1} p^{k-1}(1-p)^{i-k} = {(i-1)!\over(k-1)!(i-2k+2)!} p^{k-1}(1-p)^{i-k} \]

  The probability of the latter is again simply $p$.

  Since the two events are independent, the joint probability is the product of the two:

  \[ p_{\mathbf{k}i} = \binom{i-1}{k-1} p^{k-1}(1-p)^{i-k} \cdot p = {(i-1)!\over(k-1)!(i-2k+2)!} p^k(1-p)^{i-k}  \]

  The expected number of flips until obtaining $k$ heads is the distributed sum of
  the probability of obtaining the k$^{th}$ heads on any flip.

  \begin{align*}
    \mathbb{E}[X] 
      &= \sum_{i=k}^{\infty} i \, p_{\mathbf{k}i} \\
      &= \sum_{i=k}^{\infty} i \binom{i-1}{k-1} p^{k}(1-p)^{i-k} \\
      &= p^k \sum_{i=k}^{\infty} i \,
        \frac{(i-1)!}{(k-1)!(i-k)!} \,
        (1-p)^{i-k}
  \end{align*}

  {\color{red} TODO: finish this way of showing it. Use the formula for the
  second derivative of the geometric series.}
  
  \vspace{1cm} % space for your solution
\end{enumerate}

\section*{Exercise 1.2}

Recall the quick-select algorithm introduced in Section 1.1.3. 
The goal of this exercise is to prove that quick-select takes $O(n)$ time in expectation.
Below we present two different approaches which offer two different perspectives.
Both analyses should use linearity of expectation and we ask you to point this out explicitly.

\begin{enumerate}
  \item \textbf{Approach 1.} Analyze quick-select similarly to quick-sort, based on the sum
  of indicators $X_{ij}$.
  One approach is to reduce to a separate analysis for each of the following 4
  classes of pairs:
  \begin{enumerate}
    \item $X_{ij}$ where $i < j < k$,
    \item $X_{ij}$ where $i < k < j$,
    \item $X_{ij}$ where $k < i < j$, and
    \item $X_{ij}$ where either $i = k$ or $j = k$.
  \end{enumerate}
  For each case, show that the expected sum is $O(n)$.

  % k is the target rank

  Let $X_{ij}$ be an indicator variable that is 1 if elements of rank $i$ and
  $j$ are compared during the execution of quick-select, and 0 otherwise.

  For the four cases above, we have the following analysis:
  \begin{enumerate}
    \item $i < j < k$.
      In this case, the expected number of comparison is
      \begin{align*}
        \mathbb{E}\bigl[\sum_{i < j < k} X_{ij}\bigr] 
      &= \sum_{i < j < k} \mathbb{E}[X_{ij}] \quad \text{(by linearity of expectation)} \\
      &= \sum_{i < j < k} Pr[X_{ij} = 1] \quad \text{(since $X_{ij}$ is an indicator variable)}
      \end{align*}

      For any selected pivot of rank $p$, we have the following cases:

      \begin{enumerate}
        \item $p < i$, then the algorithm will continue and possibly compare $i$
        and $j$ in the future.
        Since only elements less than $i$ are discarded, this case only
        delays the comparison of $i$ and $j$ to a future round, and does not
        affect the analysis.  

        \item $p = i$, then $i$ and $j$ will be compared.

        \item $i < p < j$, then $i$ and $j$ will not be compared in this round,
        and one of $i$ or $j$ will be discarded, so we will never compare $i$
        and $j$ in the future.

        \item $p = j$, then $i$ and $j$ will be compared.

        \item $j < p < k$, then the algorithm will discard both $i$ and $j$, so
        we will never compare $i$ and $j$ in the future.

        \item $k = p$, then the algorithm will terminate and we will never
        compare $i$ and $j$ in the future.

        \item $p > k$, then the algorithm will continue and possibly compare $i$
        and $j$ in the future.
        Since only elements greater than $p$ are discarded, this case only
        delays the comparison of $i$ and $j$ to a future round, and does not
        affect the analysis.  

      \end{enumerate}

      Observe that $i$ and $j$ will be compared if and only if the first pivot selected among
      $\{i, i+1, \ldots, j, j+1, \ldots, k\}$ is either $i$ or $j$.

      The probability of this happening is $Pr[X_{ij} = 1] = \frac{2}{k-i+1}$.

      Thus, the expected number of comparisons is
      \begin{align*}
        \mathbb{E}\bigl[\sum_{i < j < k} X_{ij}\bigr] 
        &= \sum_{i < j < k} Pr[X_{ij} = 1] \\
        &= \sum_{i < j < k} \frac{2}{k-i+1} \\
        &= \sum_{i=1}^{k-2} \sum_{j=i+1}^{k-1} \frac{2}{k-i+1} \\
        &= 2 \sum_{i=1}^{k-2} \frac{k-i-1}{k-i+1} \\
        &= 2 \sum_{g=2}^{k-1} \frac{g-1}{g+1} \quad \text{(change of variable with $g = k-i$)}\\
        &= 2 \sum_{g=2}^{k-1} \frac{g+1-2}{g+1} \quad \text{(rewriting)}\\
        &= 2 \sum_{g=2}^{k-1} 1-\frac{2}{g+1} \quad \text{(rewriting)}\\
        &= 2 (k-2) - 4\sum_{g=2}^{k-1} \frac{1}{g+1} \\
        &= 2k-4-4\sum_{g=3}^{k} \frac{1}{g} \\
        &= 2k-4-4\sum_{g=3}^{k} \frac{1}{g} \\
        &= 2k-4-4(H_k - H_2) \quad \text{(reduce into harmonic series)}\\\\
        &= 2k-4-4(ln(k) - 1 - \frac{1}{2}) \\
        &= 2k-O(log n) \\
        &= O(k) = O(n)\\
      \end{align*}

    \item $i < k < j$.

     For this case, $i$ and $j$ is compared exactly when the first pivot
     selected among $\{i, i+1, \ldots, j\}$ is either $i$ or $j$. Note that when
     $k$ is selected, the algorithm terminates and $i$ and $j$ are not compared,
     and when any other element between $i$ and $j$ is selected, one of $i$ or
     $j$ is discarded and they will never be compared.

     Therefore, the expected number of comparisons is

     \begin{align*}
        \mathbb{E}\bigl[\sum_{i < k < j} X_{ij}\bigr] 
        &= \sum_{i < k < j} Pr[X_{ij} = 1] \\
        &= \sum_{i < k < j} \frac{2}{j-i+1} \\
        &= \sum_{i=1}^{n-2} \sum_{k=i+1}^{j-1} \frac{2}{j-i+1} \\
        &= \cdots \quad \text{(analogous to case (a))}\\
        &= O(n)
     \end{align*}

    \item $k < i < j$.
     Analogous to case (a), the expected number of comparisons is
     \begin{align*}
        \mathbb{E}\bigl[\sum_{k < i < j} X_{ij}\bigr] 
        &= \sum_{k < i < j} Pr[X_{ij} = 1] \\
        &= \sum_{k < i < j} \frac{2}{j-k+1} \\
        &= \sum_{k=1}^{j-2} \sum_{i=k+1}^{j-1} \frac{2}{j-k+1} \\
        &= \cdots \quad \text{(analogous to case (a))}\\
        &= O(n)
     \end{align*}

    \item $i = k$ or $j = k$.
      $i$ and $j$ is compared when the first pivot chosen among
      $\{i, i+1, \ldots, j\}$ is either $i$ or $j$. And we have the number of expected comparisons as:

     \begin{align*}
        \mathbb{E}\bigl[\sum_{i=k \lor j=k} X_{ij}\bigr] 
        &= \sum_{i=k \lor j=k} Pr[X_{ij} = 1] \\
        &= \sum_{i=1}^{k-1} \frac{2}{k-i+1} + \sum_{j=k+1}^{n} \frac{2}{j-k+1} \\
        &= 2\sum_{g=2}^{k} \frac{1}{g} + \quad \text{(change of variables with $g = k-i+1$)}\\
        & \qquad 2\sum_{h=2}^{n-k+1} \frac{1}{h} \quad \text{(change of variables with $h = j-k+1$)}\\
        &= O(n)
     \end{align*}

  \end{enumerate}
    
  \vspace{1cm} % space for your solution
  
  \item \textbf{Approach 2.} The following approach can be interpreted as a randomized divide
  and conquer argument. We are arguing that with constant probability, we
  decrease the input by a constant factor, from which the fast (expected) running
  time follows.
  \begin{enumerate}
    \item Consider again quick-select. Consider a single iteration where we pick a
    pivot uniformly at random and throw out some elements. Prove that with
    some constant probability $p$, we either sample the $k$th element or throw
    out at least $1/4$ of the remaining elements.

    \vspace{1cm}

    Let $X$ be a random indicator variable. 
    
    Let $X = 1$ if we pick a pivot that has a rank that is in the middle half of the entire range.
    Let $X = 0$ otherwise.

    Since the pivot is picked uniformly at random, the probability of picking a pivot
    that is in the middle half of the entire range is $Pr[X = 1] = 1/2$.

    With $X = 1$, we throw out at least either the top quarter or at least the
    bottom quarter of the whole range, which is $\geq 1/4$ of the entire range.

    Another possibility is that we find the pivot directly.

    Thus the probability of either finding the pivot or throwing out at least
    $1/4$ of the entire range is $p \geq 1/2$.

    \vspace{1cm}
    
    \item For each integer $i$, prove that the expected number of iterations (i.e.,
    rounds of choosing a pivot) of quick-select, where the number of elements
    remaining is in the range $\bigl[(4/3)^i, (4/3)^{i+1}\bigr)$, is $O(1)$.
    \vspace{1cm}

    Let $X_i$ be the number of iterations where the number of elements
    remaining is in the range $\bigl[(4/3)^i, (4/3)^{i+1}\bigr)$.

    We want to show that $\mathbb{E}[X_i] = O(1)$, that is, it is bound to a constant.

    The probability of the number of elements remaining being in the range
    $\bigl[(4/3)^i, (4/3)^{i+1}\bigr)$ after 1 iteration is the probability of
    selecting a pivot either in the first quarter or the last quarter, which is
    $1/2$.

    Thus, the expected number of iterations until the number of elements
    remaining is no longer in the range $\bigl[(4/3)^i, (4/3)^{i+1}\bigr)$ is

    The probability of the number of elements remaining in the range after $j$
    iterations is $Pr[X_i = j] = (1-p)^{j-1} p = (1/2)^j$.

    \begin{align*}
      \mathbb{E}[X_i]
      &= \sum_{j=1}^{\infty} j \, Pr[X_i = j] \\
      &= \sum_{j=1}^{\infty} j \, (\frac{1}{2})^j \\
      &= (1/2) \sum_{j=1}^{\infty} j \, (1/2)^{j-1} \\
      &= (1/2) \sum_{j=0}^{\infty} (j+1) \, (1/2)^{j} \\
      &= (1/2) \bigl(\sum_{j=0}^{\infty} j \, (1/2)^{j} + \sum_{j=0}^{\infty} (1/2)^{j}\bigr) \\
      &= (1/2) \bigl(2 + 2\bigr) \quad \text{(by formula for geometric series and its derivative)}\\
      &= 2
    \end{align*}

    Thus we have shown that $\mathbb{E}[X_i] = 2 = O(1)$.

    \vspace{1cm}
    
    \item Fix an integer $i$, and consider the amount of time spent by quick-select
    while the number of elements remaining is greater than $(4/3)^{i-1}$ and at
    most $(4/3)^i$. Show that the expected amount of time is $\leq O((4/3)^i)$.
    
    \vspace{1cm}

    Let $T_i$ be the amount of time spent by quick-select while the number of
    elements remaining is greater than $(4/3)^{i-1}$ and at most $(4/3)^i$.

    Since the pivot has to be compared with all the remaining elements, the time
    spent in each iteration is proportional to the number of remaining elements.
    Thus, the time spent in each iteration is $O((4/3)^i)$.

    \item Finally, use the preceding part to show that the expected running time of
    quick-select is $O(n)$.

    \vspace{1cm}
    Let $T$ be the total time spent by quick-select.
    Then we have 

    \begin{align*}
      \mathbb{E}[T]
      &= \mathbb{E}[\sum_{i=1}^{log_{4/3}n} T_i \, X_i] \\
      &= \sum_{i=1}^{log_{4/3}n} \mathbb{E}[T_i \, X_i] \\
      &= \sum_{i=1}^{log_{4/3}n} \mathbb{E}[T_i] \mathbb{E}[X_i] \quad \text{(by independence of $T_i$ and $X_i$)}\\
      &= \sum_{i=1}^{log_{4/3}n} ((4/3)^i) \cdot O(1) \\
      &= O(\sum_{i=1}^{log_{4/3}n} (4/3)^i) \\
      &= O(\frac{(4/3)^{log_{4/3}n + 1} - 1}{(4/3) - 1}) \quad \text{(by formula for geometric series)}\\
      &= O(\frac{(4/3)n - 1}{(4/3) - 1}) \\
      &= O(n)
    \end{align*}

    \vspace{1cm}
  \end{enumerate}
\end{enumerate}

\section*{Exercise 1.4}

This exercise is about a simple randomized algorithm for verifying
matrix multiplication. Suppose we have three $n \times n$ matrices $A, B, C$. 
We want to verify if $AB = C$. Of course one could compute the product $AB$ 
and compare it entrywise to $C$. But multiplying matrices is slow: 
the straightforward approach takes $O(n^3)$ time and there are 
(more theoretical) algorithms with running time roughly $O(n^{2.37}\ldots)$. 
We want to test if $AB = C$ in closer to $n^2$ time.

The algorithm we analyze is very simple. Select a point 
$x \in \{0,1\}^n$ uniformly at random. (That is, each $x_i \in \{0,1\}$ 
is an independently sampled bit.) Compute $A(Bx)$ and $Cx$, and compare 
their entries. (Note that it is much faster to compute $A(Bx)$ than $AB$.) 
If they are unequal, then certainly $AB \neq C$ and we output false. 
Otherwise we output true. Note that the algorithm is always correct if $AB = C$, 
but could be wrong when $AB \neq C$. We will show that if $AB \neq C$, 
the algorithm is correct with probability at least $1/2$.

\begin{enumerate}
  \item Let $y \in \mathbb{R}^n$ be a fixed nonzero vector, and let 
  $x \in \{0,1\}^n$ be drawn uniformly at random.
  Show that $\langle x, y \rangle \; \overset{\text{def}}{=} \; 
  \sum_{i=1}^n x_i y_i \neq 0$ with probability at least $1/2$.

  \vspace{1cm}

  % let y_j be a nonzero element of y
  % then x_j is 0 with probability 1/2, and 1 with probability 1/2

  % Suppose no other y_i's add up to -y_j, then if x_j is 1, then the inner
  % product is nonzero, so with probability 1/2 the inner product is nonzero.

  % Suppose there are other y_i's that add up to -y_j, such that 
  % then for the inner product to be zero,
  % Either both x_j and the other other x_i are all zero, or all of them are 1.
  % This has a probability of 1/4 + 1/4 = 1/2, in the case that there is exactly one other y_i that adds up to -y_j.
  % if x_j is 1,
  % then the inner product is nonzero, so with probability 1/2 the inner


  % if x_j is 1, then the inner product is

  Since $y$ is a fixed nonzero vector, there exists at least one $y_j \neq 0$.

  Let $y_j \neq 0$, and consider the partial sum $S_j = \sum_{i \neq j} x_i y_i$
  by sampling all $x_i$'s except $x_j$.

  For all possible $S_j$, we analyze the probability that $Pr[\langle x, y
  \rangle = 0 \vert S_j = \sum_{i \neq j} x_i y_i]$, and we will show that this
  probability is at most $1/2$ for all $S_j$. Then by the law of total
  probability, $Pr[\langle x, y \rangle = 0] <= 1/2$, thus $Pr[\langle x, y
  \rangle \neq 0] >= 1/2$.


  We split into three cases:

  \begin{enumerate}
    \item $y_j \neq -S_j$. For this case, regardless of the value of $x_j$,
    $\langle x, y \rangle = x_j y_j + S_j \neq 0$. Thus $Pr[\langle x, y
    \rangle = 0 \vert S_j] = 0 < 1/2$.

    \item $y_j = -S_j$ and $S_j \neq 0$. For this case, $\langle x, y
    \rangle = 0$ if and only if $x_j = 1$. Thus $Pr[\langle x, y \rangle = 0
    \vert S_j] = Pr[x_j = 1] = 1/2$.

    \item $y_j = -S_j = 0$. This is impossible since $y_j \neq 0$.
  \end{enumerate}

  Therefore, for all possible $S_j$, $Pr[\langle x, y \rangle = 0 \vert S_j] \leq
  1/2$. By the law of total probability, $Pr[\langle x, y \rangle = 0] \leq 1/2$.
  Thus $Pr[\langle x, y \rangle \neq 0] \geq 1/2$.

  \vspace{1cm}
  
  \item Use the preceding result to show that if $AB \neq C$, 
  then with probability at least $1/2$, $ABx \neq Cx$.

  \vspace{1cm}

  Let $D = AB - C$. Then $AB \neq C \iff D \neq 0$, and $ABx \neq Cx \iff Dx \neq 0$.

  By assumption, we can assume $D \neq 0$.

  Let $D_j$ be the row in $D$ that contains a nonzero entry, then by results
  from the previous question, we have that 
  $Pr[\langle x, D_j^{-1} \rangle = 0] \leq 1/2$. 

  Having more nonzero bits in $D$ will only decrease the probability $Pr[\langle
  x, D_j^{-1} \rangle = 0]$, since more random bits drawn in $x$ corresponding
  to the extra nonzero bits in $D$ would have to be zero, which decreases the
  total probability.

  Thus, $Pr[Dx = 0] \leq 1/2$, and by implication, we have that with probability
  at least $1/2$, $ABx \neq Cx$.
  
  \vspace{1cm}
  
  \item Suppose we want to decrease our probability of error to (say) $1/n^2$. 
  Based on the algorithm above, design and analyze a fast randomized algorithm 
  with the following guarantees:
  \begin{itemize}
    \item If $AB = C$, then it always reports that $AB = C$.
    \item If $AB \neq C$, then with probability of error at most $1/n^2$, 
    it reports that $AB \neq C$.
  \end{itemize}
  Your analysis should include the running time as well.

  \vspace{1cm}

  We will draw multiple random vectors $x_1, x_2, \cdots, x_{{\lceil 2log_2 n
  \rceil}}$, and report $AB \neq C$ if $Dx_i \neq 0$ for \textit{any} $i$ in
  $x_1, x_2, \cdots, x_{{\lceil 2log_2 n \rceil}}$.

  If $AB = C$, then $D = 0$, and we are guaranteed to have $Dx = 0$, which means
  the algorithm reports that $AB = C$.

  If $AB \neq C$, then the algorithm will report $AB = C$ only if for all $x_1,
  x_2, \cdots, x_{{\lceil 2log_2 n \rceil}}$, $Dx_i = 0$. The probability of
  this happening is equal to $(\frac{1}{2})^{\lceil 2log_2 n \rceil} =
  1/2^{\lceil 2log_2 n \rceil} \leq 1/n^2$.

  Suppose the dimension of $A$ is $n \times m$, and that of $B$ is $m \times p$,
  then the cost of computing $Bx$ is $O(mp)$; the cost of computing $A(Bx)$ is
  $O(nm)$, and the cost of computing $Cx$ is $O(np)$. For square matrices, this
  is approximately $O(n^2)$.

  The runtime of the algorithm is ${\lceil 2log_2 n \rceil} n^2 = O(n^2)$.
  
  \vspace{1cm}
\end{enumerate}

\begin{center}
{\LARGE HW0 Self–Grading Summary}\\[2mm]
{\large Graded against \textit{Appendix B: Selected Solutions} (and the provided Quick-Select solution)}
\end{center}

\section*{My Scores}
\begin{itemize}[leftmargin=1.25em]
  \item Exercise 0.10: \pts{9.5/10}
  \item Exercise 1.4: \pts{10/10}
  \item Exercise 1.2 (Quick-Select): \pts{9.5/10}
\end{itemize}

\noindent\textbf{Total:} \pts{29.0/30}

\section*{Exercise 0.10 \quad(\pts{9.5/10})}
Per Appendix~B, the rubric splits as \(5+3+2\) for parts (1)--(3).

\subsection*{(1) Expected flips until first head (\pts{5})}
I derived \(\mathbb{E}[T]=1/p\) via the series method and handled the geometric-series steps correctly. 
I did not explicitly justify exchanging differentiation and summation (or give the finite-sum limit argument), so I deduct \(0.5\) point here.\\
\textbf{Score:} \pts{4.5/5}.

\subsection*{(2) Expected flips until two heads (\pts{3})}
I modeled the waiting time as the sum of two i.i.d.\ geometric variables and used linearity of expectation to conclude \(2/p\). This matches Appendix~B.\\
\textbf{Score:} \pts{3/3}.

\subsection*{(3) Expected flips until \(k\) heads (\pts{2})}
I summed \(k\) geometric waiting times and concluded \(k/p\), consistent with Appendix~B.\\
\textbf{Score:} \pts{2/2}.

\paragraph{Subtotal:} \pts{9.5/10}.

\section*{Exercise 1.4 \quad(\pts{10/10})}
Appendix~B uses a \(4+3+3\) split for parts (1)--(3).

\subsection*{(1) \(\Pr[\langle x,y\rangle\neq 0]\ge 1/2\) for nonzero \(y\) (\pts{4})}
I conditioned on the last nonzero coordinate (partial sums argument) and showed the probability bound exactly as in the posted solution.\\
\textbf{Score:} \pts{4/4}.

\subsection*{(2) If \(AB\neq C\), then \(\Pr[ABx\neq Cx]\ge 1/2\) (\pts{3})}
I reduced to a nonzero row of \(D=AB-C\) and applied part (1) on that row, matching Appendix~B.\\
\textbf{Score:} \pts{3/3}.

\subsection*{(3) Boost to error \(\le 1/n^{2}\) in near-quadratic time (\pts{3})}
I repeated the test \(O(\log n)\) times, drove the error to \(\le 1/n^2\), and kept time near-quadratic for \(n\times n\) matrices, consistent with Appendix~B.\\
\textbf{Score:} \pts{3/3}.

\paragraph{Subtotal:} \pts{10/10}.

\section*{Exercise 1.2 (Quick-Select) \quad(\pts{9.5/10})}
I graded this using the rubric implicit in the provided Quick-Select solution (I followed \textit{Approach 2}). Either approach earns full credit when its subparts are satisfied.

\subsection*{(a) Constant-probability progress (\(\ge \tfrac12\)) (\pts{2})}
I argued that choosing a pivot from the middle half occurs with probability \(1/2\); this either finds the target or discards at least a quarter of the items.\\
\textbf{Score:} \pts{2/2}.

\subsection*{(b) Expected iterations per size band is \(O(1)\) (\pts{4})}
I partitioned by size bands \([\,(4/3)^i,(4/3)^{i+1})\), used the coin-flip intuition from Exercise~0.10 to show a good pivot in constant expected trials, and observed that a good pivot exits the band.\\
\textbf{Score:} \pts{4/4}.

\subsection*{(c) Cost per iteration in band \(i\) is \(O((4/3)^i)\) (\pts{2})}
I stated and used that each iteration within band \(i\) processes \(\Theta((4/3)^i)\) elements, so the expected work in that band is \(\Theta((4/3)^i)\cdot O(1)\).\\
\textbf{Score:} \pts{2/2}.

\subsection*{(d) Sum over bands to conclude \(\mathbb{E}[T]=O(n)\) (\pts{2})}
I summed over \(i\) using the geometric series to obtain \(O(n)\). 
I briefly wrote \(\mathbb{E}\!\left[\sum_i T_i X_i\right]\) instead of \(\sum_i \mathbb{E}[T_i]\) and did not explicitly cite linearity at that step. Since the intent and final bound are correct, I deduct \(0.5\) point.\\
\textbf{Score:} \pts{1.5/2}.

\paragraph{Subtotal:} \pts{9.5/10}.

\section*{Final Notes}
\begin{itemize}[leftmargin=1.25em]
  \item In Exercise~0.10(1), I should add a one-line justification for exchanging limit/derivative with the sum to make the argument fully rigorous.
  \item In Exercise~1.2(d), I should explicitly invoke linearity when moving the expectation inside the band-sum.
\end{itemize}

\end{document}